\documentclass{article}

\usepackage{graphicx}
\usepackage[backend=biber, style=authoryear-icomp]{biblatex}
\addbibresource{my.bib}

\newcommand{\comment}[1]{}

\begin{document}

\title{
	Software Requirement Specification \\ 
	for \\ 
	Bug-Tracker \\ 
	Version 1.0 approved
}
\author{Jesse Wood}

\maketitle 

\tableofcontents

\section{Introduction}
\subsection{Purpose}
\comment{
Identify the product whose software requirements are specified in this document, including the revision or release number. Describe the scope of the product that is covered by the SRS. Particularly if this STS describes only part of the system or a single subsytem.
	}
\subsection{Document Conventions}
\comment{
Decribe any standards of typographical conventions that were followed when writing this SRS, such as fonts or highlighting that have special significance. For exmaple, state whether priorites for higher-level requirements are assumed to be inheritied by detailed requirements, or whether every requirement statement is to have its own priorities.
	}
\subsection{Intended Audience and Reading Suggestions}
\comment{
Descibe the different types of reader that the document is inteded for, such as developers, project managers, marketing staff, users, testers, and documentation writers. Describe what the rest of the STS contains and how it is organised. Suggest a sequenc for reading the document, beginning with the overview sections and proceeding throught the sections that are most pertinent to each reader type.
}
\subsection{Product Scope}
\comment{
Provide a short decription of the software being specified and its purpose, including relevant benefits, objectives and goals. Relate the software to corporate goals or business strategies. If a separate vision and scope document is available, refer to it rather than duplicating its contents here.
}

\subsection{References}
\comment{
Provide a short description of the software being specified and its purposed, including relevant benefits, objectives and goals. Relate the software to coperate goals or business stragies. If a seperate reader could access a copy of each reference, including title, author, version number, date, and source or location.
	}
\printbibliography

\section{Overall Description}
\subsection{Product Perspective}
\comment{
Describe the context and origin of the product being specified in this SRS. For example, state whether this product is a follow-on member of a product family, a replacment for existing systems, or a new, self-contained product. If the SRS defines a component of a larger system, relate the requirments of the larger system to the functionality of this software and identift interfaces between the two. A simple diagram that shows the major components of the overall system, subsystem interconnections, and external interfaces can be helpful.
}
\subsection{Product Functions}
\comment{
	Summarize the major functions to the product must perform or must let the user perform. Details will be provided in Section 3, so only a high level summary (such as a bullet list) is needed here. Organize the functions to make them understandable to any ready of the SRS. A picture of the major groups of related requirements and how they relate, such as a top level data flow diagram or object class diagram, is often effective.
	}
\subsection{User Classes and Characteristics}
\comment{
Identify the various user classes that you anticipate will use this product. User classes may be differentiated based on the frequency of use, subset of product functions used, technical expertise, security or privelege levels, educational level, or experience. Describe the pertinent characteristics of each user class. Certain requirements may pertain only to certain user classes. Distinguish the most import user classes for this product from those who are less important to satisfy.
}
\subsection{Operating Environment}
\comment{
Describe the enviroment in which the software will operate, including the hardware platform, operating systems and versions, and any other software components or applications with which it must peacefully coexist.
	}
\subsection{Design and Implementation Constraints}
\comment{
Descirbe any items or issues that will limit the options available to developers. These might include: corporate or regulatory policies; hardware limitations (timing requirements, memory requirements); interfaces to other applications; specific technologies, tools, and databases to be used, parallel operations; language requirements; communication protocols; security considerations; design conventions or programming standards (for example, if the customer's organization will be responsible for maintaining delivered software).
}
\subsection{User Documentation}
\comment{
List the user documentation components (such as user manuals, on-line help, and tutorials) that will be delivered along with the software. Identify any known user documentation delivery formats or standards.
	}
\subsection{Assumptions and Dependencies}
\comment{
List any assumed factors (as opposed to known facts) that could affect the requirements stated in the SRS. These could include third-party or commercial components that you plan to use, issues around the development environment, or constraints. The project could be affected if these assumptions are incorrect, are not shared, or change. Also identify any dependencies the project has on external factors, such as software components that you intent to reuse from another project, unless they are already documented elsewhere (for example, in the vision and scope document or the project plan)
	}
\section{External Interface Requirements}
\subsection{User Interfaces}
\comment{
	Describe the logical characteristics of each interface between the software product and users. This may include smaple screen images, any GUI standards or product family style guides that are to be followed, screen layout contraints, standard buttons and functions (e.g., help) that will appear on every screen, keyboard shortcuts, error message display standards, and so on. Define the software components for which a user interface is needed. Details of the user interface should be documented in a seperate user interface specification.
	}
\subsection{Hardware Interfaces}
\comment{
Describe the logical and physical characteristics of each interface between the software product and the hardware components of the system. This may include the supported device types, the communication protocols to be used.
	}
\subsection{Software Interfaces}
\comment{
Describe the connections between this product and other specific software components (name and version) including databases, operating systems, tools, libraries, and integrated commercial components. Identify the data items or messages coming into the system and going out and describe the purpose of each. Describe the services needed and the nature of communications. Refer to documents that describe detailed application programmin interface protocols. Identify data that will be shared across software components. If the data sharing mechanisim must be implemented in a specific way (for example, use of a global data area in a multitasking operating system), specify this as an implementation constraint.
}
\subsection{Communication Interfaces}
\comment{
Describe the requirements associated with any communication functions required by this product, including e-mail, web-browswer, network server communication protocols, electronic forms, and so on. Define any pertinant message formatting. Identify any communication standards that will be used, such as FTP or HTTP. Specify any communication security or encryption issues, data transfer rates, and synchronization mechanisms.
}
\section{System Features}
\comment{
E.g.

Description and Priority: 
<Provide a short description of the feature and indicate whether it is of High, Medium or Low Priority. You could also include specific priority component ratings, such as benefit, penalty, cost, and risk (each rated on a relative scale from low of 1 to high of 9)>

Stimulus/Response Sequences: 
<List the sequences of user actions and system responses that stimulate behaviour defined for this feature. These will correspond to the dialog elements associated with use cases.>

Functional Requirments:
< Itemize the detailed functional requirements associated with this feature. These are the software capabilites that must be present in order for the user to carry out the services provided by the feature, or to execute the use case. Include how the product should respond to anticipated error conditions or invalid inputs. Requirements should be concise, complete, unambigious, verifiable, and necessary. Use "TBD" as a placeholder to indicate when necessary information is not yet available.>

<Each requirement should be uniquely identfied with a sequence number or a meaningful tab of some kind.>
	}
\subsection{System Feature 1}
\subsection{System Feature 2}

\section{Other Nonfunctional Requirements}
\subsection{Performance Requirements}
\comment{
If there are performance requirements for the product under various circumstances, stat them here and explain their rationale, to help developers understand the intent and make suitable design choices. Specift the timing relationships for real time systems. Make such requirements as specific as possible. You may need to state performance requirements for individual functional requirements or features.
	}
\subsection{Safety Requirements}
\comment{
Specify those requirements that are concerned with possible loss, damage, or harm that could result from use of the product. Define any safeguards or actions that must be taken, as well as actions that must be prevented. Refer to any external policies or regulations that state safety issues that affect the product's design or use. Define any safety certifications that must be satisfied.
	}
\subsection{Security Requirements}
\comment{
Specify any requirements regarding security or privacy issues surrounding use of the product or protection of the data used or created by the product. Define any user identity authentication requirements. Refer to any external policies or regulations containing security issues that affect the product. Define any security or privacy certifications that must be satisfied.
	}
\subsection{Software Quality Attributes}
\comment{
Specify any additional quality characteristics for the product that will be omportant to either the customer or the developer. Some to consider are adaptability, availability, correctness, flexibility, interoperability, maintaintability, portability, reliability, reusability, robustness, testability, and usability. Write these to be specific, quantative and verifiable where possible. At least, clarify the relative preferences for various attributes, such as ease of use over ease of learning.dd
	}
\subsection{Business Rules}
\comment{
List any principles about the product, such as which individuals or roles can perform which functions under specific circumstances. These are not functional requirments in themselves, but they may imply certain functional to enforce the rules.
	}

\section{Other Requirements}
\comment{
Define any other requirements not covered elsewhere in the SRS. This might include database requirements, internationalization requirments, legal requirments, reuse objectives for the projects, and so on. Add any new sections that are pertinent to the project.
	}	
\subsection{Gloassary}
\comment{
Define all the terms necessary to properly interpret the SRS, including acronyms and abbreviations. You may wish to build a seperate gglossary that spans multiple projects or the entire organization, and just include terms specific to a single project in each SRS.
	}
\subsection{Analysis Models}
\comment{
Optionally, include any pertinent analysis models, such as data flow diagrams, class diagrams, state-transition diagrams, or entity-relationship diagrams.
	}
\subsection{To Be Determined List}
\comment{
Collect a numbered list of the TBD (to be determined) references in the SRS so they can be tracked for closure.
	}
\end{document}
